\section{Introduction} \label{sec:introduction}
In the realm of computer vision and image processing,
the ability to accurately predict gender and age from facial
images holds significant importance across various applications that benefit
from the demographic data of their users.

Biometric information can, in fact, be used in a plethora of ways,
ranging from targeted commercial use \cite{int1} to intelligent non-profit campaigns \cite{int2}
and even extending to Orwellian credit scoring systems \cite{int3}.

A heated debate continues to unfold regarding the application
and potential abuse of Machine Learning and Computer Vision
technologies in the daily lives of citizens, particularly heightened
since the advent of Convolutional Neural Networks (CNN).
The transformative capabilities of CNNs lie in their ability to process
vast amounts of data and generate remarkably accurate predictions.
Notably, these networks eliminate the necessity for manual feature
engineering tasks, such as feature extraction, thereby rendering the
implementation and utilization of these technologies more convenient than
ever before \cite{int6}. 

This evolution raises significant questions about privacy, ethics, and
the broader societal impact of seamlessly integrating advanced algorithms
into various aspects of our lives \cite{int7}. 
Notably, entities such as the European Union Commission and Parliament
have actively addressed these concerns by formulating the AI Act \cite{int4}.
This legislative initiative seeks to classify the risks associated with
ML and CV technologies, especially concerning the citizens of the
confederation. The primary objective is to safeguard individuals
from the inappropriate use of their biometric data, acknowledging
the critical need to establish regulatory frameworks that balance
the advancement of technology with the protection of individual
rights and privacy  \cite{int5}.

Given the circumstances, it is crucial to understand the design
techniques and architectures upon which these models are based
to utilize and implement them with increased awareness and
consideration for the effects and consequences on end-users.
In particular, we will undertake a comparison between a multi-task
CNN and two single-task CNNs for
age and perceived gender detection to assess their results and
differences in task execution.