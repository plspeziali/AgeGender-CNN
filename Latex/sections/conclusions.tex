\section{Conclusion} \label{sec:conclusions}
In this study, we have presented a comparison between two CNN
architectures for the recognition of biometric characteristics,
specifically age and gender, from facial images.

As evident from the metrics and discernible in various graphical
representations, the binary gender classification task achieves similar
results in both architectures. In contrast, the age regression task
attains superior performance when the network is trained for both tasks,
maintaining an identical CNN structure. The observed improvement suggests
that joint training for multiple tasks, in this case, age regression and
gender classification, positively influences the model's ability to predict
age compared to the scenario where only a single task is addressed.

Certainly, while the current observations highlight the advantageous
impact of joint training on age regression performance compared
to a single-task approach, it's essential to acknowledge that these
conclusions may be subject to variation under different conditions
and with the implementation of alternative techniques.
The effectiveness of multi-task learning can be influenced by various
factors such as dataset characteristics and model architecture.