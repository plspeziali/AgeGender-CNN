\documentclass[letterpaper, 10 pt, journal, twoside]{IEEEtran} 
% Use this command for final RAL version
%
\usepackage{amsmath}
\usepackage{bm}
\usepackage{breqn}
\usepackage{amssymb}
\usepackage{subcaption}
\usepackage{graphicx}
\usepackage{color}
\usepackage[usenames,dvipsnames,svgnames,table]{xcolor}
\usepackage{tabularx}
\usepackage{multirow}
\usepackage{url}
\usepackage{enumitem}   
\usepackage{xcolor}
\usepackage{makecell}
\usepackage{adjustbox}
\usepackage{booktabs}

%\overrideIEEEmargins
% Comment this command for final RAL version.
% Use this command for initial and revised RAL versions, and for final conference version

\newcommand\inputfile[1]{%
  \InputIfFileExists{#1}{}{\typeout{No file #1.}}%
}
\newcommand{\todo}[1]{\noindent\textbf{\textcolor{red}{[TODO: #1]}}}


\makeatletter
\DeclareRobustCommand\icgonedot{\futurelet\@let@token\@icgonedot}
\newcommand\@icgonedot{\ifx\@let@token.\else.\null\fi\xspace}
\newcommand\eg{\textit{e.g., }}
\newcommand\Eg{\textit{E.g., }}
\newcommand\ie{\textit{i.e., }} 
\newcommand\Ie{\textit{I.e., }}
\newcommand\ia{\textit{i.a., }}
\newcommand\Ia{\textit{I.a. ,}}
\newcommand\cf{\textit{cf}} 
\newcommand\Cf{\textit{C.f.}}
\newcommand\etc{\textit{etc.}}
\newcommand\vs{\textit{vs.}}
\newcommand\wrt{w.r.t.}
\newcommand\aka{a.k.a.}
\newcommand\dof{d.o.f.}
\newcommand\etal{et al.}
\newcommand\OFC{\textrm{OFC}\xspace}

%Added by thomas
\newcommand{\mbf}[1]{\mathbf{#1}}
\newcommand{\xb}{\mbf{x}}
\newcommand{\Xb}{\mbf{X}}
\newcommand{\yb}{\mbf{y}}
\newcommand{\zb}{\mbf{z}}
\newcommand{\hb}{\mbf{h}}
\newcommand{\ub}{\mbf{u}}
\newcommand{\Wk}[1]{\mbf{W}_k^{#1}}
\newcommand{\xbp}{\mbf{x^\prime}}
\newcommand{\txb}{\mbf{\tilde{x}}}
\newcommand{\txbp}{\mbf{\tilde{x}^\prime}}
\newcommand{\T}[1]{#1^\mathsf{T}}
\newcommand{\RR}{I\!\!R}
\newcommand{\Mbb}{\mathbb{M}}
\newcommand{\Obb}{\mathbb{O}}
\newcommand{\Zbb}{\mathbb{Z}}
\newcommand{\Ybb}{\mathbb{Y}}
\newcommand{\Rbb}{\mathbb{R}}


\makeatother

%\newcommand\eg{\textit{e.g}\icgonedot}
\begin{document}

\title{Project Title}


\author{Author Name Author Surname}


\maketitle

\begin{abstract}
This study conducts a comparative analysis between a multi-task
Convolutional Neural Network (CNN) and two single-task CNNs for
age and perceived gender detection in facial images, aiming to evaluate their
performance and differences in task execution.
The two single-task CNNs were designed with identical structures,
differing only in the final part: one for age prediction and another
for gender classification. The multi-task CNN incorporated a branching
mechanism for both tasks, utilizing a shared loss function for backpropagation.
Evaluation metrics included R-squared score for age prediction,
accuracy for gender
classification, and visualization of mean heatmaps to discern disparities in
the attention of convolutional layers. Results indicate that there are no
significant differences in performance between the multi-task CNN and the
single-task CNNs, suggesting comparable efficacy in gender and age prediction
tasks.

\end{abstract}

\IEEEpeerreviewmaketitle
\section{Introduction} \label{sec:introduction}
In the realm of computer vision and image processing,
the ability to accurately predict gender and age from facial
images holds significant importance across various applications that benefit
from the demographic data of their users.

Biometric information can, in fact, be used in a plethora of ways,
ranging from targeted commercial use \cite{int1} to intelligent non-profit campaigns \cite{int2}
and even extending to Orwellian credit scoring systems \cite{int3}.

A heated debate continues to unfold regarding the application
and potential abuse of Machine Learning and Computer Vision
technologies in the daily lives of citizens, particularly heightened
since the advent of Convolutional Neural Networks (CNN).
Notably, these networks eliminate the necessity for manual feature
engineering operations, such as feature extraction, thereby rendering the
implementation and utilization of these technologies more convenient than
ever before 
in image processing and recognition tasks
\cite{int6}. 

This evolution raises significant questions about privacy, ethics, and
the broader societal impact of seamlessly integrating advanced algorithms
into various aspects of our lives \cite{int7}. 
Notably, entities such as the European Union Commission and Parliament
have actively addressed these concerns by formulating the AI Act \cite{int4}.
This legislative initiative seeks to classify the risks associated with
ML and CV technologies, especially concerning the citizens of the
confederation. The primary goal is to safeguard individuals
from the inappropriate use of their biometric data, acknowledging
the critical need to establish regulatory frameworks that balance
the advancement of technology with the protection of individual
rights and privacy  \cite{int5}.

Given the circumstances, it is crucial to understand the design
techniques and architectures upon which these models are based,
to utilize and implement them with increased awareness and
consideration for the effects and consequences on end-users.
In particular, we will undertake a comparison between a multi-task
CNN and two single-task CNNs for
age and perceived gender detection to assess their results and
differences in task execution.
\section{Related Work} \label{sec:related_work}
Several advancements have been made in the field of gender
and age prediction from facial images, utilizing deep learning techniques.
In this section, we present two notable works that have contributed
significantly to the state-of-the-art in this domain.

The work by [Rafique et al., 2019] \cite{rel1} introduces a deep learning
framework based on an ensemble of attentional and residual convolutional
networks with the primary objective of predicting gender and age groups
with a high accuracy rate, treating both features as a classification problem.
The proposed model, trained on the UTKFace dataset,
employs attention mechanisms to focus on
crucial facial regions, enhancing the accuracy of predictions. 
The multi-task learning approach is utilized, and the feature embedding
of the age classifier is augmented with predicted gender information.

In a different approach, [Antipov et al., 2017] \cite{rel2} explores
improvements in existing CNN-based methods for gender and age prediction.
The study investigates key factors that impact the training of
CNNs, including target age encoding, loss function,
CNN depth, pretraining necessity, and training strategy
(mono-task or multi-task).
The authors present state-of-the-art gender recognition and
age estimation models designed according to benchmarks such as
LFW, MORPH-II, and FG-NET.
Notably, their best model won the ChaLearn Apparent Age
Estimation Challenge 2016, significantly
outperforming the solutions of other participants.
\section{Proposed Approach} \label{sec:approach}
 
\subsection{Dataset and Preprocessing} \label{sec:dataset}

The dataset we will be utilizing is the UTKFace dataset,
which consists of $23,708$ aligned and cropped facial images, 
annotated with age, gender, and ethnicity labels.
This dataset was created with the intention of covering
a wide range of variations, including pose, facial expression,
illumination, occlusion, resolution, and more.
In our analysis, we will specifically concentrate
on the first two attributes within the dataset: age and gender. 

In our analysis, we will exclusively consider $70\%$
of the dataset for our training set, with the remaining
$30\%$ designated for testing purposes.
This division allows us to train our models on a substantial
portion of the data while maintaining a separate,
untouched set for rigorous evaluation.

The training set exhibits an age distribution depicted
in the histogram
shown in Fig.~\ref{1age}, along with a balanced gender distribution,
as illustrated in Fig.~\ref{2gender1}.

\begin{figure}[htbp]
    \centerline{\includegraphics[width=.5\textwidth]{images/dataset/age.png}}
    \caption{Histogram of age distribution in the training set}
    \label{1age}
\end{figure}

\begin{figure}[htbp]
    \centerline{\includegraphics[width=.4\textwidth]{images/dataset/gender1.png}}
    \caption{Histogram of gender distribution in the training set}
    \label{2gender1}
\end{figure}

We further split the training set into a $70\%$ training set and a $30\%$
validation set. The validation set plays a critical role in refining
and making our model more robust, preventing overfitting through techniques
such as early stopping.

Despite this subdivision, we have taken measures to maintain the
balance in the dataset: the accompanying histogram in  illustrates that,
concerning gender, the distribution remains similar in both datasets.
As for age, a continuous value that poses challenges for balance
assessment, we will employ the non-parametric Kolmogorov-Smirnov test
to scrutinize the absence of statistically significant differences in
the distributions of its values between the two datasets.
This rigorous approach ensures the reliability and fairness of our
evaluation process.

\begin{figure}[htbp]
    \centerline{\includegraphics[width=.5\textwidth]{images/dataset/gender2.png}}
    \caption{Histogram of gender distribution in the new training
    set and the validation set}
    \label{3gender2}
\end{figure}

The outcome of the aforementioned test yields a $p$-value of approximately
$0.38$. This value is reasonably high,
leading us to consider it sufficient evidence to accept the null hypothesis.
Thus, we conclude that the two distributions of age in the two
datasets do not exhibit statistically significant differences.

The concluding step in our dataset preprocessing involves the application
of data augmentation techniques. Specifically, we expand the dataset
by incorporating horizontally mirrored images and introducing random
rotations of up to 10 degrees. Additionally, we employ random adjustments
to the brightness, contrast, saturation, and hue of the images.
This augmentation strategy has been employed based on empirical
evidence suggesting that enlarging the dataset enhances the model's
performance. By introducing these variations in orientation and color,
we aim to expose the model to a more diverse set of examples,
ultimately improving its ability to generalize and make accurate
predictions on unseen data.

\subsection{Model Architecture} \label{sec:model}

As previously mentioned, in this project, we will compare two
CNN architectures, one comprising a single multi-task
network and another consisting of two single-task networks.
The overarching goal is common between them.

Let's delve into the details of the first architecture.

\subsubsection{Multi-task Network} \label{sec:multi}


\section{Experiments} \label{sec:experiments}

\subsection{Training Environment and Hardware Specifications}

The model training was conducted within a Python Notebook environment
in Anaconda, utilizing PyTorch libraries with GPU acceleration.
The machine used features an AMD Ryzen 7 5800X CPU with 8 cores and
16 threads, operating at 4 GHz and
it is equipped with 16 GB of RAM running at 3600 MHz
and a PNY NVidia RTX 3070 graphics card with 8GB
of VRAM and 5888 CUDA cores.

\subsection{Training Process}

The model training process utilized a total of seconds,
as indicated in the following table:
\begin{table}[H]
    \centering
    \begin{tabular}{@{}lll@{}}
    \toprule
    Time (sec)               & \textbf{Age task} & \textbf{Gender task} \\ \midrule
    \textit{Single-task CNN} &  1223.41          & 747.39               \\
    \textit{Multi-task CNN}  &  640.34           & *                    \\ \bottomrule
    \end{tabular}
\end{table}

Additionally, the epoch count for each training network is presented below:
\begin{table}[H]
    \centering
    \begin{tabular}{@{}lll@{}}
    \toprule
    Num. of epochs           & \textbf{Age task} & \textbf{Gender task} \\ \midrule
    \textit{Single-task CNN} & 17                & 10                   \\
    \textit{Multi-task CNN}  & 9                 & *                    \\ \bottomrule
    \end{tabular}
\end{table}

In each training iteration, an early stopping mechanism was implemented with
a patience value set to 3 for both architectures.
The patience hyper-parameter was determined by monitoring the validation loss
of the model and stopping the training process when the loss
did not improve for a number of epochs equal to the patience value.

The training process was halted when the loss values
on the training set reached: 
\begin{table}[H]
    \centering
    \begin{tabular}{@{}lll@{}}
    \toprule
    Last loss value (training) & \textbf{Age task} & \textbf{Gender task} \\ \midrule
    \textit{Single-task CNN} &  0.7951           & 0.0027               \\
    \textit{Multi-task CNN}  &  1.2228           & *                    \\ \bottomrule
    \end{tabular}
\end{table}
and on the validation set:
\begin{table}[H]
    \centering
    \begin{tabular}{@{}lll@{}}
    \toprule
    Last loss value (validation) & \textbf{Age task} & \textbf{Gender task} \\ \midrule
    \textit{Single-task CNN} &  1.3360           & 0.0041               \\
    \textit{Multi-task CNN}  &  1.4058           & *                    \\ \bottomrule
    \end{tabular}
\end{table}
During the backpropagation, the loss value for the age task, 
was normalized by multiplying it by a factor
$\lambda_{\text{age}} = 0.01$.
By scaling down the loss for the age task, it effectively equalized
the magnitudes of both losses, ensuring a more balanced and comparable
optimization process. 

In Fig.~\ref{1loss}, Fig.~\ref{2loss} and Fig.~\ref{3loss},
we can observe the trends through the epochs
of the loss
in the single task for age, the single task for gender
and the multi-task scenario, respectively.
In Fig.~\ref{4acc} and Fig.~\ref{5acc}, we can observe the
trends through the epochs of the accuracy
in the single task
and the multi-task scenario, respectively.

\begin{figure}[htbp]
    \centerline{\includegraphics[width=.45\textwidth]{images/training/loss-single-age.png}}
    \caption{Loss trend in the single task for age}
    \label{1loss}
\end{figure}
\begin{figure}[htbp]
    \centerline{\includegraphics[width=.45\textwidth]{images/training/loss-single-gender.png}}
    \caption{Loss trend in the single task for gender}
    \label{2loss}
\end{figure}
\begin{figure}[htbp]
    \centerline{\includegraphics[width=.45\textwidth]{images/training/loss-multi.png}}
    \caption{Loss trend in the multi task}
    \label{3loss}
\end{figure}

\begin{figure}[htbp]
    \centerline{\includegraphics[width=.45\textwidth]{images/training/acc-single.png}}
    \caption{Accuracy trend in the single task CNNs}
    \label{4acc}
\end{figure}
\begin{figure}[htbp]
    \centerline{\includegraphics[width=.45\textwidth]{images/training/acc-multi.png}}
    \caption{Accuracy trend in the multi task CNN}
    \label{5acc}
\end{figure}

The training process ended with the following metrics on the training set:
\begin{table}[H]
    \centering
    \begin{tabular}{@{}lll@{}}
    \toprule
    Training Metrics & \textbf{Age task ($\mathbf{R^2}$)} & \textbf{Gender task (Accuracy)} \\ \midrule
    \textit{Single-task CNN} & 0.8664            & 0.9278               \\
    \textit{Multi-task CNN}  & 0.7955            & 0.9112               \\ \bottomrule
    \end{tabular}
\end{table}
and the following on the validation set:
\begin{table}[H]
    \centering
    \begin{tabular}{@{}lll@{}}
    \toprule
    Validation Metrics & \textbf{Age task ($\mathbf{R^2}$)} & \textbf{Gender task (Accuracy)} \\ \midrule
    \textit{Single-task CNN} & 0.7693            & 0.8978               \\
    \textit{Multi-task CNN}  & 0.7546            & 0.9020               \\ \bottomrule
    \end{tabular}
\end{table}
We can observe how similar results are achieved with both the training
and validation set. It is important to note that the true measure
of the performance of the models will be revealed in the testing set
and we will only then analyze if an architecture has an advantage
over the other.

In the context of the provided sample image in Fig.~\ref{1heat},
taken from the validation set,
we can observe the corresponding mean attention heatmaps
for each convolutional layer in the neural network.
Specifically, Fig.~\ref{2heat} illustrates the single-task attention
heatmap for age, while Fig.~\ref{3heat} depicts the single-task
attention heatmap for gender and
Fig.~\ref{4heat} showcases the multi-task attention heatmap.
We can observe that the attention maps in the multi-task
scenario appear to be a weighted average,
with a notable bias towards the age-related features,
as compared to the two individual single-task scenarios.

\begin{figure}[htbp]
    \centerline{\includegraphics[width=.19\textwidth]{images/heatmaps/sample.png}}
    \caption{Sample image from the validation dataset}
    \label{1heat}
\end{figure}
\begin{figure}[htbp]
    \centerline{\includegraphics[width=.36\textwidth]{images/heatmaps/h_age.png}}
    \caption{Attention heatmap for the Age single-task CNN}
    \label{2heat}
\end{figure}
\begin{figure}[htbp]
    \centerline{\includegraphics[width=.36\textwidth]{images/heatmaps/h_gender.png}}
    \caption{Attention heatmap for the Gender single-task CNN}
    \label{3heat}
\end{figure}
\begin{figure}[htbp]
    \centerline{\includegraphics[width=.36\textwidth]{images/heatmaps/h_multi.png}}
    \caption{Attention heatmap for the multi-task CNN}
    \label{4heat}
\end{figure}

\subsection{Training Result}

After completing the model training, we can now transition to the testing
phase. During this phase, we load the testing images, which were initially
set aside without undergoing any preprocessing, and evaluate
the capabilities of the CNNs in explaining that data.
Below, you can find the table of accuracy and $R^2$ results for both
architectures:
\begin{table}[H]
    \centering
    \begin{tabular}{@{}lll@{}}
    \toprule
    Test Metrics & \textbf{Age task ($\mathbf{R^2}$)} & \textbf{Gender task (Accuracy)} \\ \midrule
    \textit{Single-task CNN} & 0.7832            & 0.8965               \\
    \textit{Multi-task CNN}  & 0.7731            & 0.8986               \\ \bottomrule
    \end{tabular}
\end{table}
along with Fig.~\ref{6mat} and Fig.~\ref{7mat} illustrating the
confusion matrices for single-task
and multi-task classification, respectively.
\begin{figure}[htbp]
    \centerline{\includegraphics[width=.45\textwidth]{images/testing/mat_single.png}}
    \caption{Confusion Matrix for the single task}
    \label{6mat}
\end{figure}
\begin{figure}[htbp]
    \centerline{\includegraphics[width=.45\textwidth]{images/testing/mat_multi.png}}
    \caption{Confusion Matrix for the multi task}
    \label{7mat}
\end{figure}
To provide a visual example of the performance of the regressor,
we present, in Fig.~\ref{8reg} and Fig.~\ref{9reg},
a graph depicting the predicted age versus the ground truth
for both the single-task and multi-task scenarios for the first
50 observations of the set.
\begin{figure}[htbp]
    \centerline{\includegraphics[width=.45\textwidth]{images/testing/reg_single.png}}
    \caption{Regressor performance example for the single task}
    \label{8reg}
\end{figure}
\begin{figure}[htbp]
    \centerline{\includegraphics[width=.45\textwidth]{images/testing/reg_multi.png}}
    \caption{Regressor performance example for the multi task}
    \label{9reg}
\end{figure}
\section{Conclusion} \label{sec:conclusions}
In this study, we have presented a comparison between two CNN
architectures for the recognition of biometric characteristics,
specifically age and gender, from facial images.

As evident from the metrics and discernible in various graphical
representations, the binary gender classification task achieves similar
results in both architectures. In contrast, the age regression task
attains superior performance when the network is trained for both tasks,
maintaining an identical CNN structure. The observed improvement suggests
that joint training for multiple tasks, in this case, age regression and
gender classification, positively influences the model's ability to predict
age compared to the scenario where only a single task is addressed.

Certainly, while the current observations highlight the advantageous
impact of joint training on age regression performance compared
to a single-task approach, it's essential to acknowledge that these
conclusions may be subject to variation under different conditions
and with the implementation of alternative techniques.
The effectiveness of multi-task learning can be influenced by various
factors such as dataset characteristics and model architecture.


%\begin{acknowledgements}
%If you'd like to thank anyone, place your comments here
%and remove the percent signs.
%\end{acknowledgements}

% BibTeX users please use one of
%\bibliographystyle{spbasic}      % basic style, author-year citations
%\bibliographystyle{spmpsci}      % mathematics and physical sciences
\bibliographystyle{IEEEtran}       % APS-like style for physics
\bibliography{bibliography}   % name your BibTeX data base


\end{document}
% end of file template.tex

